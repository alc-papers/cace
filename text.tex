\doublespacing{}
\section{Introduction.}
\paragraph{}
Should companies be allowed to track consumers' shopping or other preferences
without their permission? This is a topic of international debate,
between multinational corporations, and privacy advocacy groups. There are
numerous reasons why companies should not be able to track consumers, from
it being a 4th amendment violation, to there being a risk of consumer data
being leaked or stolen. There have been numerous times where companies have
had data stolen from them because of lack of proper security. Therefore
Companies should not be able to spy on consumers' shopping or other
preferences without their permission.
\par

\section{The 4th amendment and data mining.}
\paragraph{}
Companies spying on consumers without their consent is a violation of a persons
4th amendment right. The 4th amendment states; ``[t]he right of the people to
be secure in their persons, houses, papers, and effects, against unreasonable
searches and seizures, shall not be violated, and no Warrants shall issue, but
upon probable cause, supported by Oath or affirmation, and particularly
describing the place to be searched, and the persons or things to be seized.''
\cite{cornell20164th}. What this means is that under no circumstance, besides
having a warrant or probable cause shall a citizen's belongings be searched
through, or seized (taken). This could be interpreted in
a manner in which the surveillance and data mining done my companies today
could be seen as a seizing of a consumer's personal belongings. Also the
monetization of consumer data (acquired by sometimes illegal means)
can potentially hurt the constitution, but consumers themselves
by putting them in direct risk, which happened with the following in August
of 2015.
\begin{quote}
The FTC said Sequoia One and Gen X Marketing Group, which both
primarily operated out of Florida, supplied Ideal Financial with account
information from at least 500,000 people who applied for payday loans, leading
to more than \$7 million being taken from those consumers' bank accounts
without their consent. The consumers learned about the charges only `after
Ideal Financial had debited their bank accounts, the complaint said.'~\cite{andreapeterson2015}
\end{quote}
The blatant irresponsibility of the data brokerage firm lead to millions of
dollars of consumer money being stolen, and crippling private information being
leaked, a strong reminder of what happens when greedy companies disregard
consumer privacy and safety. A third example of an instance where the 4th
amendment could be broken is with a relatively creepy piece of technology, store
mannequins. ``\ldots{}Even the store mannequins have gotten in on the gig. According
to the \textit{Washington Post}, mannequins in some high-end boutiques are now
being outfitted with cameras that utilize facial recognition technology. A small
camera embedded in the eye of an otherwise normal looking mannequin allows to
keep track of the age, gender, and race of all their customers.''~\cite{johnw.whitehead2012}.
Again we face interpretation problems with the 4th constitution upon what
``effects'' are in context to the modern world. If one considers his or her
personal information and data ``effects'', then  this technology would be
breaking their 4th amendment constitutional right. Though, with that said, it is
not like companies could not provision methods of ensuring customers with peace
of mind with their data, by doing things like having wavers and updating privacy
policies so that they are not ridiculously long, and are actually readable.
\par

\section{Consumer data could be stolen or leaked.}
\paragraph{}
The incompetence of corporations has been proved in the past, with various
incidents in which companies have lost, leaked, or had consumer data outright stolen
from them. A very infamous instance being the PSN (Play Station Network) outage
of 2011. ``The 2011 PlayStation Network outage was the result of an ``external
intrusion'' on Sony's PlayStation Network and Qriocity services, in which
personal details from approximately 77 million accounts were compromised\ldots{}''
\cite{sonypr2011}\cite{bbc2011}\cite{shanerichmond2011}\cite{chrisgriffith2011}
The vulnerability of data in the digital age is high, even when guarded by the
biggest of corporations. On top of that, identifying these intrusions isn't easy
as explained by Sony to the PSN users. ``There’s a difference in timing between
when we identified there was an intrusion and when we learned of consumers’ data
being compromised. We learned there was an intrusion April 19th and subsequently
shut the services down. We then brought in outside experts to help us learn how
the intrusion occurred and to conduct an investigation to determine the nature
and scope of the incident. It was necessary to conduct several days of forensic
analysis, and it took our experts until yesterday to understand the scope of the
breach. We then shared that information with our consumers and announced it
publicly this afternoon.''\cite{patrickseybold2011}. Furthermore inadequate
security measures nonchalantly done by companies can pose a huge risk as yet
again\ldots{}example by Sony in their outage incident. ``Credit card data was
encrypted, but Sony admitted that other user information was not encrypted at
the time of the intrusion.''\cite{sonyqa2011}
\cite{keithstuartcharlesarthur2011} The Daily Telegraph reported that ``If the
provider stores passwords unencrypted, then it's very easy for somebody else---
not just an external attacker, but members of staff or contractors working on
Sony's site---to get access and discover those passwords, potentially using
them for nefarious means.''\cite{christopherwilliams2011} On May 2, Sony
clarified the ``unencrypted''status of users' passwords, stating that:
\cite{patrickseybold20113}
While the passwords that were stored were not ``encrypted,'' they were
transformed using a cryptographic hash function. There is a difference between
these two types of security measures which is why we said the passwords had not
been encrypted. But I want to be very clear that the passwords were not stored
in our database in cleartext form.''\,''.
\par

\section{Consumers can not stop this tracking.}
\paragraph{}
Whether consumers want to or not, they are not able to stop the merciless
tracking done by hounding enterprises without an armory of browser add-ons,
and programs. ``The ubiquitous blue ``Like'' or ``Share'' buttons that you see
all over the Internet are hiding an ugly secret. Starting this month, Facebook
will use them to track your visit to every Web page that displays the buttons—
even if you don’t click on anything. Facebook will use the data it collects to
build a detailed dossier of your browsing habits\ldots{}''\cite{natecardozo2015}
The seemingly harmless
like button is now like an eye of ``big brother'' on every page it is on,
watching you everywhere you go, tracking all you without your consent.
Though spying isn't limited to just unwitting consumers, even students
are being pried on, most without their knowledge! As quoted later in the
same article. ``Google’s Chromebooks as used in schools also come with ``Chrome
Sync'' enabled by default, a feature that sends the student users’ entire
browsing trail to Google, linking the data collected to the students’ accounts
which often include their names and dates of birth.''\cite{natecardozo2015}
This is in direct contradiction with one of google's OWN policies, ``That’s true
even despite Google’s signature on the `Student Privacy Pledge' which
includes a commitment to `not collect\ldots{}student personal information beyond that
needed for authorized educational/school purposes, or as authorized by the
parent/student.' ''~\cite{natecardozo2015}. ``Do Not Track'' buttons are
just as big of a fail as the prevention of the prevention of unwanted
telemarketing calls. In most cases opting into ``Do Not Track'' options
makes you a bigger target for companies as it sends special code to the website
when you browse the web. The FTC (Federal Trade Commission) is responsible for
this program, and their failure is saddening. ``But the strategy was flawed from
the start. By tapping the World Wide Web Consortium, an organization that sets
standards for the Web, to work out the details for implementing Do Not Track,
the FTC relied on a group dominated by powerful Internet companies. These
companies included Google, Facebook and Yahoo, whose businesses depend on online
advertising, which require precision tracking of users. To put it another way,
that’s like Sony Pictures inviting the North Koreans to run vulnerability tests
on its computer networks.''~\cite{dawnchmielewski2016}. To put trust in the same
companies who were violating peoples' privacy in the first place was a terrible
mistake on the FTCs part. Government organizations fail their citizens by
not giving them the proper tools they need to stop tracking, or just lazily
leaving it to a third party.
\par

\newpage
\printbibliography{}
